\documentclass[twocolumn,10pt]{article}
\usepackage[margin=1in]{geometry}
\usepackage[T1]{fontenc}
\usepackage[sort,numbers]{natbib}
\usepackage{graphicx}
\usepackage{siunitx}
\usepackage{amsmath,xspace}
\usepackage[backgroundcolor=yellow]{todonotes}
\usepackage[utf8]{inputenc}
\usepackage{bibentry}
\usepackage{float}
\usepackage[ttscale=.875]{libertine}
\usepackage{listings}
\usepackage{libertinust1math}
\usepackage{mdframed}
\usepackage[format=plain,labelfont={bf,it},textfont=it]{caption}
\usepackage[breaklinks]{hyperref}

\author{Daniel Bittman \\ dbittman@ucsc.edu}

\setlength{\columnsep}{7mm}

\title{\sesh: \textbf{Implementing Re-establishable\\Sessions Without a Session Layer}
\\
{\vspace{5mm}\normalsize CMPE252A Final Project\\\vspace{-3mm} Professor: Chen Qian}
}

\newcommand{\etal}{\emph{et~al.}\xspace}
\newcommand{\unix}{\textsc{Unix}\xspace}
\newcommand{\sesh}{\texttt{sesh}\xspace}

\begin{document}
\biolinum
\maketitle
\libertine
\renewcommand\ttdefault{lmtt}

\section*{Abstract}


\begin{center}\noindent\rule{2cm}{0.4pt}\end{center}

The source-code for \sesh, along with raw data and generation scripts,
can be found at \url{https://github.com/dbittman/sesh}.

\section{Introduction}





\section{Design}



\section{Implementation}



\section{Microbenchmarks}


\begin{figure}
	\centering
	\includegraphics[width=70mm]{fig/time_est}
	\caption{Latency for establishing a TCP connection using \sesh. The time
	represents the additional overhead of the session functionality over that of
	a normal TCP connection establishment.}
	\label{fig:est}
\end{figure}

\begin{figure}
	\centering
	\includegraphics[width=70mm]{fig/time_recon}
	\caption{Latency for reconnecting with a new TCP connection between a client
	and a server with a previously established session.}
	\label{fig:recon}
\end{figure}




\section{Related Work}

Providing ``mobile IP'' services is common in cell networks, since users are
largely truly mobile and so need a built-in method to have changing IP addresses
are part of the network design~\cite{ltemob,mobileip,Kurose}. While this largely
solves the problem for cell networks, it does not provide interoperability for
cell-phones that reconnect to WiFi networks, which do not have the same
functionality. This limitation is indicative of a larger limitation---the
network must support the requirements of a mobile-IP infrastructure. This
contrary to arguments in~\cite{Saltzer}, because it requires support through the
whole stack. Furthermore, these solutions often require support by client
applications and potentially servers (which may or may not understand the
complexities of mobile-IP services, resulting in errant behavior).
OpenFlow~\cite{McKeown} also
provides features for mobility with flows, but also requires extensive support
in the hardware and network.

Other projects and papers, such as Chandrashekar's Service Oriented
Internet~\cite{chandrashekar2003service} and Saif's Service-Oriented Network
Sockets~\cite{Saif} discuss designs for, essentially, session layer network designs.
While \sesh would eventually provide more session-like features than
simple reconnect and session tokens, that work is outside the scope of a class
project. However, much of the work in the aforementioned papers is extremely
relevant to the future of a library such as \sesh.

A session layer protocol that explicitly provides reconnection is
\textit{fived}~\cite{wasptr-15-01}. \textit{Fived} runs as a daemon on both
endpoints, through which clients and servers can establish sessions.
\textit{Fived} provides a number of services that \sesh does not yet,
but should in the future, support, such as service enumeration. However,
\textit{fived} incurs additional overhead by running as a daemon, whereas
\sesh runs as libraries that interpose library calls, improving
performance.

Numerous projects focus on problem-avoidance for reconnecting dead TCP
connections~\cite{mosh,autossh,screen,tmux}. Solutions range from implementing
sessions on the server (which require forethought and explicit use of tools), to
reparenting processes to continue running on hangup (which is not a session), to
reestablishing a new connection automatically (which is also not a session). All
these solutions lack the usability and transparency of true sessions.

\section{Conclusion}













\bibliography{sesh}
\bibliographystyle{plainnat}

\end{document}

